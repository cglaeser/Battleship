\documentclass[a4paper]{article}

\usepackage[utf8]{inputenc}
\usepackage[ngerman,english]{babel}

%opening
\title{Dokumentation - TTV Praktikum 3\&4}
\author{Allers, Sven \& Gläser, Christian}

\begin{document}

\maketitle

\begin{abstract}

\end{abstract}

\section{Initialisierung}
Aus der Konsole aufrufen mit folgenden Parametern:
programname[]

\section{Spielstrategie}
\subsection{Auswahl des Opponenten für das Ziel}
(Zu finden in: StatisticsManager-Killselector)\newline
Alle Spieler haben zum Spielbeginn die gleiche Anzahl Schiffe. \newline
Ausgewählt wird in jeder Runde der Gegner, der die meisten Anzahl an getroffenen Schiffen hat.\newline
Bei gleicher Anzahl an getroffener Schiffe wird als nächstes Kriterium die Anzahl der nicht getroffenen Felder berücksichtigt.\newline
Bei gleicher Anzahl an nicht getroffenen Feldern wird die Nachbarschaft berücksichtigt, die Nähe im Uhrzeigersinn ist ein eindeutiges Selektionskriterium.
\subsection{Auswahl des Feldes}
\begin{enumerate}
	\item Startfelder der Spieler werden geschätzt. Grundlage: Empfangene Nachrichten
		\subitem Spieler werden nach ID sortiert
		\subitem Startfeld = ID des Vorgängers + 1
		\subitem Endfeld = ID des Spielers
	\item Berechnung: Welche logischen Felder wurden getroffen
	\item Berechnung: Zufällige Auswahl eines noch nicht getroffenen Feldes
	\item In die Mitte des Feldes, welches aus mehreren Chord-IDs besteht wird ein Schuss abgesetzt.
\end{enumerate}
\end{document}
